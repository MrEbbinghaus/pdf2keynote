%----------------------------------------------------------------------------------
%
% beamer2keynote slides
%
%----------------------------------------------------------------------------------

\documentclass{beamer}

\usetheme{metropolis}

\usepackage[latin1]{inputenc}
\usepackage{verbatim}
\usepackage{pgfpages}

%\setbeamertemplate{note page}[plain] % to simplify the task of extracting notes
\setbeameroption{show notes on second screen=right}

\title{PDF to Keynote}
\subtitle{and using Beamer to generate Keynote presentations}
\author{R�my M�ller}

\begin{document}

%----------------------------------------------------------------------------------

\begin{frame}
    \titlepage
    \note{Notes for title page}
\end{frame}

%----------------------------------------------------------------------------------

\begin{frame}[fragile]
    \frametitle{Usage}  
    Usage:
    \begin{verbatim}
        pdf2keynote pdf_path [-o keynote_path] 
    \end{verbatim}

If your pdf is generated from Beamer with the option
\begin{verbatim}
\usepackage{pgfpages}
\setbeameroption{show notes on second screen=right}
\end{verbatim}
Then the text notes from the second screen (the right half of the pdf page) are extracted as presenter notes in Keynote.

    \note
    {
        Notes for usage slide \\
        with multiline \\
        comments
    }
\end{frame}

%----------------------------------------------------------------------------------

\begin{frame}
    \frametitle{A Slide Without Notes}  
    Just an empty slide without notes
\end{frame}

\begin{frame}[fragile]
    \frametitle{A Frame with itemized notes}    
    One can also use rich text for comments, for example
    \begin{verbatim}
    \note {\begin{itemize}
            \item one 
            \item two
            \item three
        \end{itemize}
        $\nabla H(x) = \partial H / \partial x$
    }
    \end{verbatim}
    \note
    {
        \begin{itemize}
            \item one 
            \item two
            \item three
        \end{itemize}
        $\nabla H(x) = \partial H / \partial x$
    }
\end{frame}

%----------------------------------------------------------------------------------

\begin{frame}
    \frametitle{Conclusion} 
    This is it!
    \note{Notes for Conclusion}
\end{frame}


\end{document}
